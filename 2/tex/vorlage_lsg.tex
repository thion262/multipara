% ===============================================
% Sogenannte Praeambel: Seiteneinstellungen etc.
% ===============================================

\documentclass[german,10pt,oneside, fleqn, a4paper]{article}

%%% Eingebundene Pakete: werden hier nicht alle gebraucht, schaden aber auch nicht.
\usepackage{graphicx}
\usepackage[german]{babel}
\usepackage{amsmath,amsthm,amssymb,amsfonts,amscd,amsbsy,amsxtra}
\usepackage{epsfig,color}
\usepackage{subfigure}
\usepackage{fancyheadings}
\usepackage{psfrag}
\usepackage{amsmath}
\usepackage{enumerate}
\usepackage[T1]{fontenc} 
\usepackage{ucs}
\usepackage[utf8]{inputenc}

%%% Neudefinition von Kommandos, um sich Tipp-Arbeit zu ersparen 
\newcommand {\Q}	{\mathbb{Q}}
\newcommand {\R}	{\mathbb{R}}
\newcommand {\N}	{\mathbb{N}}
\newcommand{\Ra}	{\Rightarrow}
\newcommand{\La}	{\Leftarrow}
\newcommand{\LRa}{\Leftrightarrow}

%%% Veraenderung des Seitenlayouts
\setlength{\oddsidemargin}{-1cm}
%\setlength{\evensidemargin}{0cm}
\setlength{\textwidth}{18cm}
\setlength{\textheight}{26cm}
\setlength{\parindent}{0pt}
\setlength{\topmargin}{-2cm}
\pagestyle{empty}

\sloppy

% ==============================================
% Start des eigentlichen Dokumentes
% ==============================================
\begin{document}


Till Fischer\\

    
\vspace{1cm}

\begin{center}
{\bf \Large Labor Softwareentwurf mit Multiparadigmen-Programmiersprachen} \\[2ex]
\end{center}
\begin{center}
{\bf \Large Versuch 2} \\[1ex]
\end{center}
\vspace{2cm}

{\bf Versuch 2-1} \\[2ex]
\begin{enumerate}[a)]
	\item Die Deklaration sagt dem Compiler dass das Objekt (Variable, Funktion) existiert und beschreibt diesen, um es dem Compiler moeglich zu mcahen, dieses zu referenzieren.
	In der Definition wird das Objekt schlussendlich instanziert/implementiert. Dies ist noetig, damit der Linker Referenzen zu den Objekten anlegen kann.
	\item Zur sicheren Referenz von Methoden ueber mehrere Quelldateien hinweg und zur Vermeidung von mehrfacher Funktionsdefinition.
	\item Zum forcieren einer Inline expansion. Dies dient der Verbesserung der Laufzeit und Access Time auf kosten der Programmgroesse. Ein weiterer Vorteil ist die statische Typpruefung.
	\item Eine Initialisierungsliste ist eine Moeglichkeit, Parameter im Konstruktor zu initialisieren, die ohne explizite Zuweisung auskommt. Ausserdem kann man damit structs im Constructor verwenden.
	\item Funktionsueberladung ist die mehrfache Dekaration einer Funktion mit jeweils verschiedenen Typen und/oder Anzahl an Argumenten. Je nachdem mit welchen Argumenten diese aufgerufen wird, wird jeweils die passende Funktion ausgewaehlt.
\end{enumerate}

{\bf Versuch 2-2} \\[2ex]
\begin{enumerate}[b]
	\item -c in gcc kompiliert nur. Eine solche Binaerdatei ist noch nicht mit Libraries verlinkt, dies geschieht beim L    inken. Er bewirkt, dass beide so kompilierte Dateien in den Compiler eingegeben werden, um Linking zu ermoeglichen.
	\item \begin{itemize}
		\item vermeidet es variablen in den globalen namespace zu setzen
		 \item man muss nun zum adressieren der Variablen ausserhalb des namespaces einen prefix verwenden.
		 \item ermoeglicht es teile der standard-lib zu ueberlagern
		 \item es gibt keine access specifier wie private oder public
	\end{itemize}
\end{enumerate}

{\bf Versuch 2-3} \\[2ex]
\begin{itemize}
	\item std::qsort benoetigt einen Funktionspointer auf die Vergleichsmethode, um eine Sortierung durchzufuehren.
	\item Nach
\end{itemize}


\end{document}
