% ===============================================
% Sogenannte Praeambel: Seiteneinstellungen etc.
% ===============================================

\documentclass[german,10pt,oneside, fleqn, a4paper]{article}

%%% Eingebundene Pakete: werden hier nicht alle gebraucht, schaden aber auch nicht.
\usepackage{graphicx}
\usepackage[german]{babel}
\usepackage{amsmath,amsthm,amssymb,amsfonts,amscd,amsbsy,amsxtra}
\usepackage{epsfig,color}
\usepackage{subfigure}
\usepackage{fancyheadings}
\usepackage{psfrag}
\usepackage{amsmath}
\usepackage{enumerate}
\usepackage[T1]{fontenc} 
\usepackage{ucs}
\usepackage[utf8]{inputenc}

%%% Neudefinition von Kommandos, um sich Tipp-Arbeit zu ersparen 
\newcommand {\Q}	{\mathbb{Q}}
\newcommand {\R}	{\mathbb{R}}
\newcommand {\N}	{\mathbb{N}}
\newcommand{\Ra}	{\Rightarrow}
\newcommand{\La}	{\Leftarrow}
\newcommand{\LRa}{\Leftrightarrow}

%%% Veraenderung des Seitenlayouts
\setlength{\oddsidemargin}{-1cm}
%\setlength{\evensidemargin}{0cm}
\setlength{\textwidth}{18cm}
\setlength{\textheight}{26cm}
\setlength{\parindent}{0pt}
\setlength{\topmargin}{-2cm}
\pagestyle{empty}

\sloppy

% ==============================================
% Start des eigentlichen Dokumentes
% ==============================================
\begin{document}


Till Fischer\\

    
\vspace{1cm}

\begin{center}
{\bf \Large Labor Softwareentwurf mit Multiparadigmen-Programmiersprachen} \\[2ex]
\end{center}
\begin{center}
{\bf \Large Versuch 1} \\[1ex]
\end{center}
\vspace{2cm}

{\bf Versuch 1-3} \\[2ex]
\begin{enumerate}[a)]
	\item 
	\begin{itemize}
		\item Prozedurale Programmierung: \\Aneinanderhaengen von Anweisungen, die dann Prozeduren ergeben.
		\item Objektorientierte Programmierung: \\Gruppierung von Funktionen, Daten etc. in Objekte und zugehoerige Funktionen.
		\item Generische Programmierung: \\Typen allgemein spezifizieren und erst bei Ausfuehrung festlegen, fuehrt zu mehr Wiederverwendbarkeit von Code.
		\item Funktionale Programmierung: \\Anweisungen als mathematische Funkionen angeben.
	\end{itemize}
	\item
		Ein \textit{struct} ist ein zusammengesetzter Datentyp,welcher aus mehreren primitiven Datentypen besteht.\\
		\textit{const} macht die Variable / Pointer zu seiner linken unveraenderbar.
	\item Der Praeprozessor in C/C++ ist ein Textersetzungsprogramm, um Makros/Direktiven in C/C++ Code umzuwandeln. Makros bergen das Problem, dass sie nicht durch die Syntax- Semantikanalyse des Programmcodes erfasst werden und die Fehlersuche schwieriger wird.
	\item Eine Referenz (\&) gibt die Adresse einer Variable zurueck. Ein Pointer (*) dereferenziert eine Variable und gibt den Inhalt der Referenz zurueck.
	\item Preprocessing(Include Dateien zusammensuchen), Compilation(Kompilieren was benoetigt wird), Linking(alles zusammenfuegen).
	
\end{enumerate}



{\bf Versuch 1-4} \\[2ex]
\begin{enumerate}[b)]
	\item 
		\begin{itemize}
			\item {\bf{g++:}} Aufruf des Compilers
			\item {\bf{-std=c++11:}} Auswaehlen des C Standards
			\item {\bf{-Wall:}} Alle Warnings
			\item {\bf{-Wextra:}} Zusaetzliche Warnings, die nicht in Wall enthalten sind.
			\item {\bf{-o hello.out:}} Benennung der Output-Datei
			\item {\bf{hellp-world.cc:}} Angabe, was kompiliert werden soll.
		\end{itemize}
\end{enumerate}

{\bf Versuch 1-7} \\[2ex]
Man muss math.h inkludieren.


{\bf Versuch 1-8} \\[2ex]
\begin{itemize}
	\item Probleme: Bei c), Angaben der Arraylaenge in der Funktionsdefinition, kann man laengere Arrays uebergeben.
	 \item Probleme: Arrays koennen in C nur durch Call-by-Reference uebergeben werden. sizeof gibt daher nur die laenge des Pointers in byte aus.
	 Daher kann man wie bei d) die Laenge des Arrays fest definieren, oder die Arraylaenge mit angeben.

\end{itemize}

\end{document}
